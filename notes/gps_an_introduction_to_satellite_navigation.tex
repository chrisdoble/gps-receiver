\documentclass{article}

\usepackage{amsmath}
\usepackage{bookmark}
\usepackage{siunitx}

\hypersetup{
  colorlinks=true,
  linkcolor=blue,
  urlcolor=blue
}

\begin{document}

This document contains my notes on Stanford's online course \href{https://www.youtube.com/playlist?list=PLGvhNIiu1ubyEOJga50LJMzVXtbUq6CPo}{GPS: An Introduction to Satellite Navigation}. Each section corresponds to the video of the same title.

\tableofcontents

\section{GPS How and Why}

\begin{itemize}
  \item In order to calculate the receiver's position we need to know:

        \begin{enumerate}
          \item the time at which a satellite transmitted a radio signal,

          \item the location of the satellite when it transmitted the signal,

          \item the speed of the radio transmission (close to the speed of light), and

          \item the time at which the radio signal is received.
        \end{enumerate}

  \item If we can obtain these four pieces of information from at least four satellites, we can solve an equation for four unknowns: the offset of the user's clock from the satellites' clocks, and the user's x, y, and z coordinates.

  \item The offset of the user's clock from the satellites' clocks is a single unknown rather than one for each satellite because all the satellites' clocks are synchronised.
\end{itemize}

\section{Satellites}

\begin{itemize}
  \item GPS satellites are in medium Earth orbit (MEO).

  \item A single GPS satellite can typically see one third of the Earth's surface.

  \item There are additional satellites in geostationary orbit (GEO) above various countries to augment GPS data.
\end{itemize}

\section{Navigation Messages}

\begin{itemize}
  \item The navigation message tells us the location of the satellite and the time at which it broadcast the navigation message.

  \item An \textbf{ephemeris} is the orbital data for a satellite.

  \item This information is broadcast by each satellite at around $\qty{50}{bps}$.

  \item A full GPS message consists of 25 pages. Each page consists of 5 subframes. Each subframe is $\qty{300}{bits}$. Thus, it takes $\qty{6}{s}$ to transmit a subframe, $\qty{30}{s}$ to transmit a page, and $\qty{12.5}{min}$ to transmit a message.

  \item Each page consists of information about the broadcasting satellite, \\ ephemeris parameters, and a page of the almanac.

  \item The ephemeris parameters are expressed as Keplerian elements. These can be split into three categories:

        \begin{itemize}
          \item The first describes the shape of the elliptical orbit itself. It doesn't position the ellipse relative to the Earth. This category includes:

                \begin{itemize}
                  \item the semi-major axis $a$ which determines the size of the ellipse, and

                  \item and the eccentricity $e$ which determines how circular or elliptical the orbit is.
                \end{itemize}

                GPS orbits are close to circular ($e = 0$), but not quite. To increase accuracy we must account for the eccentricity of the orbit.

          \item The next describes how the orbit is oriented relative to the Earth. The Earth is positioned at one of the foci of the ellipse. This category includes:

                \begin{itemize}
                  \item the inclination $i$ which is the angle the orbital plane makes with the equatorial plane,

                  \item the right ascension of the ascending node (RAAN) $\Omega$ which is the angle between the vernal equinox and the ascending node of the orbit in the equatorial plane in the direction of the Earth's rotation, and

                  \item the angle of perigee $\omega$ which is the angle between the ascending node and perigee in the orbital plane.
                \end{itemize}

          \item The last describes the satellite's position in the orbit. This category contains only the true anomaly $\nu$ which is the angle between perigee and the satellite in the orbital plane.
        \end{itemize}
\end{itemize}

\section{Navigation Signals}

\begin{itemize}
  \item There is a unique code for each satellite.

  \item Each $0$ or $1$ in a satellite's code is known as a \textbf{chip}.

  \item Satellites transmit at $\qty{1.023}{Mcps}$ (million chips per second).

  \item The L1 frequency is the most used civilian frequency.

  \item The code for each satellite has good autocorrelation properties (i.e. it's easy to see when the receiver has aligned its code with the transmitted code) and low cross-correlation with other satellites.
\end{itemize}

\end{document}