\documentclass{article}
\usepackage{amsmath}
\usepackage{bookmark}

\hypersetup{
  colorlinks=true,
  linkcolor=blue,
  urlcolor=blue
}

\begin{document}

This document contains my notes on Stanford's online course \href{https://www.youtube.com/playlist?list=PLGvhNIiu1ubyEOJga50LJMzVXtbUq6CPo}{GPS: An Introduction to Satellite Navigation}. Each section corresponds to the video of the same title.

\tableofcontents

\section{GPS How and Why}

\begin{itemize}
  \item In order to calculate the receiver's position we need to know:

        \begin{enumerate}
          \item the time at which a satellite transmitted a radio signal,

          \item the location of the satellite when it transmitted the signal,

          \item the speed of the radio transmission (close to the speed of light), and

          \item the time at which the radio signal is received.
        \end{enumerate}

  \item If we can obtain these four pieces of information from at least four satellites, we can solve an equation for four unknowns: the offset of the user's clock from the satellites' clocks, and the user's x, y, and z coordinates.

  \item The offset of the user's clock from the satellites' clocks is a single unknown rather than one for each satellite because all the satellites' clocks are synchronised.
\end{itemize}

\section{Satellites}

\begin{itemize}
  \item GPS satellites are in medium Earth orbit (MEO).

  \item A single GPS satellite can typically see one third of the Earth's surface.

  \item There are additional satellites in geostationary orbit (GEO) above various countries to augment GPS data.
\end{itemize}

\end{document}